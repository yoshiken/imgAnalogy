% abstract.tex -- 論文概要
現行IntrusionDet tionSystem (IDS)にはアノマリ型とシグネチャ型が存在している。
パターンファイルをセッティングするシグネチャ型は既存攻撃には強いが未知攻撃や亜種攻撃に対しては検知できない課題がある。
そこで未知攻撃や亜種攻撃に対して検出するためのアノマリ型がIDSが注目を集めている。
しかし実環境において稼働に耐える学習データを生成するデータ量を確保するには多大な時間的コストがかかることが問題となっている。
本研究では類似するネットワークからキャプチャーしたネットワークデータを用いて学習にかかる時間的コストの削減を提案する。
より実環境に近い状態を想定して脆弱性スキャナ及び実ネットワークのトラフィックを用いて評価を行う。
なお、検知対象としてHTTPにおける攻撃に注目しており、プロトコルはHTTP限定する。
