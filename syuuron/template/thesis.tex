% thesis.tex -- 論文の書き方参考例
%
% (注意)氏名、学籍番号等を変更すること。
%
% (LaTeXの実行法) platex thesis.tex
%
%	このファイル内にある '% 'はコメント。
%
\documentclass[a4j,11pt]{jreport}
\usepackage{ascmac}
\usepackage{amsmath}
\usepackage{float}
\usepackage{url}
\usepackage[dvipdfmx]{graphicx}
\usepackage{thesis-master}
\usepackage{mylatex}
\pagestyle{plain}

% 所属研究科、専攻
\courseofmastercs
% 研究テーマ
\title{類似トラフィックを用いた機械学習の初期学習にかかるコスト軽減}
% 氏名
\author{吉田健太}
% 学籍番号
\id{G2116032}
% 指導教員
\teacher{木下 俊之}
% 提出日
\date{2018}{1}{23}
% 提出年度
\schoolyear{20XX}

% 研究室名(カバー用)
\clab{木下}
% 学籍番号(カバー用)
\cid{G2116032}

%% 英語要旨
\etitle{Title in English}
\eauthor{T~a~r~o~~K~o~u~k~a}
\eteacher{D~a~i~j~i~r~o~~K~o~u~k~a}

\begin{document}
\makecover		% カバー
\maketitle		% 表紙

% 修士論文日本語概要
\jabst{
% abstract.tex -- 論文概要
近年電子計算機の処理能力向上により、システムの大規模化が進んでいる。だがシステムの規模が大きくなるほど、システムデータやバックアップデータが増大しストレージの圧迫することになってしまう。\\
そうした自体に対して圧縮を行いデータ量を減らす手法があるが、従来の圧縮技術では計算コストや管理の面倒などが問題になっている。そこでデータを個々で見る従来の手法に代わるストレージ全体を見て重複している部分の排除を行う重複排除(Dedup:Data De-deplication)が開発された。\\
本研究では、画像ファイルを用いてDedupによる重複排除の前に画像の類似性でグループ分けをし、より効率的な重複排除方法を提案する。
	% abstract.texを読み込む
}
\makejabstract

% 修士論文英語概要
\eabst{
% abstract.tex -- 論文概要
Write an abstract of your Paper.

}
\makeeabstract

% 目次
\pagenumbering{roman}
\setcounter{page}{1}
\thispagestyle{plain}
\tableofcontents	% 目次
\listoffigures		% 図目次
\listoftables		% 表目次
%\listofprograms		% プログラム目次

% 論文本文
%   論文本文は章や節単位で,いくつかのファイルに分割したほうが編集しやすい.
% \input{hogehoge}	% hogehoge.texを読み込む.
% 章構成は適宜変更すること.
\pagenumbering{arabic}
\chapter{はじめに}
\section{はじめに}
近年、インターネットが爆発的に普及し、個人間だけではなく企業間の取引・軍事的にも重要な役割を担っている。だが、普及と同時に多数の脅威も現れた。\\
その脅威の一部がマルウェアと呼ばれる。マルウェアの定義とは文脈によって様々であるが、不正かつ有害な動作をするもの、自己増殖を行い感染を拡大するものなどが代表的な定義とされている。\\
マルウェアの歴史は古く、まだインターネットが一部の大学・企業間のみの通信の時代から生成された。その後様々なマルウェアが作られ、OS開発元やセキュリティソフトを開発してる企業などはいたちごっこを繰り返してきた。\\
インターネットにおける脅威はマルウェアだけではなく、ハッキング・クラッキングによる脅威も依然猛威を振るっている。\\
それらの脅威の行動源となっているものとして、初期は自己のハッカーとしての技術を自慢する承認欲求や感染対象の反応をみて快感を得る愉快犯が多かったが、近年では敵対企業の信用を落としたりランサムウェアによる身代金を得たりする利益を得ようとする傾向が出ている。\\
以上のようにインターネットの普及率や技術の向上から様々な脅威が発生する中、各国ではサーバー空間を第5の戦場として対応しているというのが現状である。\\

\section{研究の背景}
脅威の対応は初期からあまり変わらず、侵入に対するパターンマッチを侵入口に設置してあるFirewall(FW)・Intrusion Detection System(IDS)にその脅威が発生するネットワークかパターンをセットし、それらで検知した通信を脅威をみなし検知・拒否を行う。\\
そういったパターンはセキュリティアナリスト達の手動で生成されていたが、脅威の発生からパターン作成・配布のタイムラグによるゼロデイ攻撃が後を絶たなかった。だが2000年前半からマルウェアの検体などを機械学習にかけることによってパターンを自動生成する研究が進み実用化されたことによって、パターン生成の速度が上がっており、そういった攻撃のリスクを減らすことが可能となった。\\
だが、これらの対策は後手に回ってしまい既存の攻撃には強くとも、未知の攻撃もしくは既存の攻撃にオリジナリティを加えた亜種攻撃に対しては脆弱であった。\\
そして、近年上記のようなパターンマッチを主とするシグネチャ型から、通常の通信を正常な通信とみなしそれ以外の通信を以上とみなすアノマリ型の研究が進んでいる。\\
だが、アノマリ型はシグネチャ型の特質と相反するものであり、未知攻撃・亜種攻撃には強くとも既存の攻撃には弱いという点が問題として挙げられている。更に、アノマリ型は正規の通信のデータの蓄積が精度と比例するため、大量の学習データが必要となり時間的コストがかかってしまうという点がある。\\


\section{研究の目的}
本研究ではアノマリ型IDSに対して時間的コストの削減を目指す。\\
近年では様々な手法によってこの時間的コストや精度向上を目指す研究が行われているが、本研究ではそういったアルゴルズム的な視点ではなく、技術が進化しても対応できる時代的視点が長く使えるようなシステムの構築を目的とする。\\
		% 第1章 はじめに
\chapter{関連技術}

\section{重複除外}

\subsection{シングルスタック}

\subsection{固定長ブロック法}

\subsection{可変長ブロック法}


\section{Hash}

\subsection{SHA-1}

\subsection{MD5}

\section{OpenCV}

\section{image File}

\subsection{フォーマット}
		% 第2章 関連技術
\chapter{提案}
\section{システムの全体説明}
\subsection{正規通信の収集・加工方法}
\subsection{学習データの統合}
\section{クライアント側での運用}
	% 第3章 提案
\chapter{実装}
章立ては指導教員の方針に従ってください.
	% 第4章 実装
\chapter{例: 評価}
章立ては指導教員の方針に従ってください.
	% 第5章 評価
\chapter{評価}
\section{結果概要}
	% 第6章 おわりに
% acknowlegments.tex -- 謝辞
\theacknowledgments
本論文の作成にあたり、終始適切な助言を賜り、また丁寧に指導して下さった木下 俊之先生にこの場を借りて感謝の意を表します。
		% 謝辞
\bibliographystyle{junsrt}
\bibliography{mybib}
%
% 自分の名前の下に下線を引いても良いかもしれません
%

\begin{theachievement}{99}
  \bibitem{EC2015} 岡田昌浩, 井上亮文, 星徹, ``投映面の特性を3DCGに反映させるシステムSUNDIALの色認識機能への拡張'', 情報処理学会研究報告, Vol.2015-EC-35, No.16, pp.1-7, 2015.
  \bibitem{NERU} 松尾豊, ``なぜ私たちはいつも締め切りに追われるのか'', \url{http://ymatsuo.com/papers/neru.pdf}, 2015年6月閲覧.
\end{theachievement}
% 業績
% appendix.tex -- 付録
\appendix
\chapter{ソースコード}
\section{CONTENT}
リストファイルを\pgref{pg:CONTENT}に示す.

\lstinputlisting[caption=CONTENT,
label=pg:CONTENT]{00CONTENT}
	% 付録
\end{document}
