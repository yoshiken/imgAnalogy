\chapter{関連技術}
\section{IDS}
IDSとはIntrusion Detection System、つまりは不正侵入検知システムの略称である。\\
IDSの機能として、ネットワーク上に流れるパケットを解析し、不正なアクセスの痕跡を見つけた場合管理者に通報するというものである。\\
通常IDSは内部ネットワークと外部ネットワークとの境界に置き、両間の通信を読み取り解析するためサイバーセキュリティの要とも言えるシステムである。\\
似た機能を搭載したシステムとしてFW・IPSが存在する。\\
FWは通常IPアドレス・プロトコル・ポート番号のみをチェックし、中身を見ることができないため通常のアクセスと偽装したアクセスとの区別がつくことができない。例えばweb観閲のためのPort80は通す設定にしていると、マルウェア通信でPort40通信している場合に検知することができない。\\
IPSはIDSと機能自体はほぼ同じなのだが、大きく違う点はIPSは異常を検知して管理者に通報するのみだが、IDSはアクセス自体を遮断するということである。IPSとIDSはセキュリティポリシーや設置場所によって使い分けることが重要となる。\\


\section{scikit-learn}
近年機械学習開発が積極的に進んでおり、様々なアルゴルズムが使用できるよう、ライブラリの開発が進んでいる。\\
scikit-learnはそのような複数のライブラリをまとめて使いやすくしたライブラリである。
サポートベクターマシン、ランダムフォレスト、Gradient Boosting、k近傍法、DBSCANなどを含む様々な分類、回帰、クラスタリングアルゴリズムを備えており、Pythonの数値計算ライブラリのNumPyとSciPyとやり取りするよう設計されている。

\section{metasploit}

\section{Nessus}
