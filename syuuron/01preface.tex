\chapter{はじめに}
\section{はじめに}
近年、インターネットが爆発的に普及し、個人間だけではなく企業間の取引・軍事的にも重要な役割を担っている。だが、普及と同時に多数の脅威も現れた。\\
その脅威の一部がマルウェアと呼ばれる。マルウェアの定義とは文脈によって様々であるが、不正かつ有害な動作をするもの、自己増殖を行い感染を拡大するものなどが代表的な定義とされている。\\
マルウェアの歴史は古く、まだインターネットが一部の大学・企業間のみの通信の時代から生成された。その後様々なマルウェアが作られ、OS開発元やセキュリティソフトを開発してる企業などはいたちごっこを繰り返してきた。\\
インターネットにおける脅威はマルウェアだけではなく、ハッキング・クラッキングによる脅威も依然猛威を振るっている。\\
それらの脅威の行動源となっているものとして、初期は自己のハッカーとしての技術を自慢する承認欲求や感染対象の反応をみて快感を得る愉快犯が多かったが、近年では敵対企業の信用を落としたりランサムウェアによる身代金を得たりする利益を得ようとする傾向が出ている。\\
以上のようにインターネットの普及率や技術の向上から様々な脅威が発生する中、各国ではサーバー空間を第5の戦場として対応しているというのが現状である。\\

\section{研究の背景}
脅威の対応は初期からあまり変わらず、侵入に対するパターンマッチを侵入口に設置してあるFirewall(FW)・Intrusion Detection System(IDS)にその脅威が発生するネットワークかパターンをセットし、それらで検知した通信を脅威をみなし検知・拒否を行う。\\
そういったパターンはセキュリティアナリスト達の手動で生成されていたが、脅威の発生からパターン作成・配布のタイムラグによるゼロデイ攻撃が後を絶たなかった。だが2000年前半からマルウェアの検体などを機械学習にかけることによってパターンを自動生成する研究が進み実用化されたことによって、パターン生成の速度が上がっており、そういった攻撃のリスクを減らすことが可能となった。\\
だが、これらの対策は後手に回ってしまい既存の攻撃には強くとも、未知の攻撃もしくは既存の攻撃にオリジナリティを加えた亜種攻撃に対しては脆弱であった。\\
そして、近年上記のようなパターンマッチを主とするシグネチャ型から、通常の通信を正常な通信とみなしそれ以外の通信を以上とみなすアノマリ型の研究が進んでいる。\\
だが、アノマリ型はシグネチャ型の特質と相反するものであり、未知攻撃・亜種攻撃には強くとも既存の攻撃には弱いという点が問題として挙げられている。更に、アノマリ型は正規の通信のデータの蓄積が精度と比例するため、大量の学習データが必要となり時間的コストがかかってしまうという点がある。\\


\section{研究の目的}
本研究ではアノマリ型IDSに対して時間的コストの削減を目指す。\\
近年では様々な手法によってこの時間的コストや精度向上を目指す研究が行われているが、本研究ではそういったアルゴルズム的な視点ではなく、技術が進化しても対応できる時代的視点が長く使えるようなシステムの構築を目的とする。\\
